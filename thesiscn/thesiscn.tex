%!TEX program = xelatex
%!encoding = utf-8
% 完整编译: xelatex -> bibtex -> xelatex -> xelatex
\documentclass[lang=cn,9pt,a4paper,cite=authoryear]{lewisthesis}

% ======================================================================
%
%
%   LiBoyu的蜜汁作业模板
%
%
% ======================================================================
% TODO:
%  - 修改文档标题
%  - 删除正文内容
%  - 编写文档


\title{作业模板}
\author{2053642 李博宇}
\institute{计算机科学与技术}
\version{1.0}
\date{\zhtoday}

% fontspec:字体修改支持,默认为宋体/TimesNewRoman
\usepackage{fontspec}
\newcommand{\ccr}[1]{\makecell{{\color{#1}\rule{1cm}{1cm}}}}
% lil:lstinline缩写
\newcommand{\lil}[1]{\lstinline{#1}}
% spc:快捷插图,语法:\spc{\path\to\pic.png}{caption},默认0.7倍行宽度
\newcommand{\spc}[2]{\begin{figure}[H]\centering\includegraphics[width=0.7\linewidth]{#1}\caption{#2}\end{figure}}

\allowdisplaybreaks

\begin{document}

\maketitle

\begin{abstract}
这里是摘要。
\keywords{作业,模板,\LaTeX{}}
\end{abstract}

\tableofcontents

\newpage

\section{第一级段落}

\subsection{第二级段落}

\subsubsection{第三级段落}

\paragraph{段落}

段落内容。

\subparagraph{子段落}

子段落内容。

\section{更丰富的内容}

插图的语法参考此处源码附近的注释。
% figure语法:
% \begin{figure}[H] ... \end{figure}

\begin{lstlisting}[style=lstypy,caption={这是一段\lil{Python}代码}]
def f(x):
	strlit = "literal"
  ans = 1
  # comments
  for i in range(1,x+1):
    ans = ans * i
  return ans
\end{lstlisting}

\begin{multicols}{2}
\begin{lstlisting}[style=lstycxx,caption={这是一段分栏\lil{C/C++}代码}]
#include<bits/stdc++.h>

int main()
{
	std::cout<<"Hello World!"<<std::endl;
  // comments
  /* comments */
  0;
  return 0;
}
\end{lstlisting}
\end{multicols}

\begin{table}[H]
\centering
\caption{这里是表头}
\begin{tabular}{cc}
\hline
字段1 & 字段2 \\
\hline
记录11 & 记录12 \\
记录21 & 记录22 \\
\hline
\end{tabular}
\end{table}

\end{document}
