%!TEX program = xelatex
%!encoding = utf-8
% 完整编译: xelatex -> bibtex -> xelatex -> xelatex
\documentclass[lang=en,9pt,a4paper,cite=authoryear]{lewisthesis}

% ======================================================================
%
%
%   LiBoyu的蜜汁作业模板
%
%
% ======================================================================
% TODO:
%  - 修改文档标题
%  - 删除正文内容
%  - 编写文档


\title{Thesis Template}
\author{2053642 Li Boyu}
\institute{Computer Science, CEIE, Tongji Univ.}
\version{1.0}
\date{\today}

% fontspec:字体修改支持,默认为宋体/TimesNewRoman
\usepackage{fontspec}
\newcommand{\ccr}[1]{\makecell{{\color{#1}\rule{1cm}{1cm}}}}
% lil:lstinline缩写
\newcommand{\lil}[1]{\lstinline{#1}}
% spc:快捷插图,语法:\spc{\path\to\pic.png}{caption},默认0.7倍行宽度
\newcommand{\spc}[2]{\begin{figure}[H]\centering\includegraphics[width=0.7\linewidth]{#1}\caption{#2}\end{figure}}

\allowdisplaybreaks

\begin{document}

\maketitle

\begin{abstract}
You can fill the content of the abstract or simply delete it as you please.
\keywords{homework, thesis, template, \LaTeX{}}
\end{abstract}

\tableofcontents

\newpage

\section{Section A}

\subsection{Subsection A.1}

\subsubsection{Subsubsection A.1.1}

\paragraph{Paragraph A.1.1.a}

This is the content of Paragraph A.1.1.a.

\subparagraph{Subparagraph A.1.1.a.a}

This is the content of Subparagraph A.1.1.a.a.

\section{Richer context}

The following \TeX{} code allows you to insert a figure to the article.

\begin{lstlisting}[language=TeX]
% Adding a figure to the article:
% (The two methods are equivalent)
\begin{figure}[H] ... \end{figure}
\spc{path}{caption}
\end{lstlisting}

\begin{lstlisting}[style=lstypy,caption={\lil{Python} Source Code}]
def f(x):
	strlit = "literal"
  ans = 1
  # comments
  for i in range(1,x+1):
    ans = ans * i
  return ans
\end{lstlisting}

\begin{multicols}{2}
\begin{lstlisting}[style=lstycxx,caption={\lil{C/C++} Source Code}]
#include<bits/stdc++.h>

int main()
{
	std::cout<<"Hello World!"<<std::endl;
  // comments
  /* comments */
  0;
  return 0;
}
\end{lstlisting}
\end{multicols}

\begin{table}[H]
\centering
\caption{Table Header}
\begin{tabular}{cc}
\hline
Field1 & Field2 \\
\hline
Record11 & Record12 \\
Record21 & Record22 \\
\hline
\end{tabular}
\end{table}

\end{document}
